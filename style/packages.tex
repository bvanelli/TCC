

%***************************** PACOTES *********************************** %
%\usepackage[within=none]{newfloat}
\usepackage[utf8]{inputenc}		     % Codificacao do documento (conversão automática dos acentos)


%	Fonte
\usepackage[T1]{fontenc}		   	 % Selecao de codigos de fonte.
\usepackage[brazil]{babel}
\usepackage{lmodern}			     % Usa a fonte Latin Modern

%	Matemático e Gráfico
\usepackage{float}
\usepackage{graphics,graphicx}	     % pacotes para inserir figuras .eps ou .jpg
\graphicspath{{figuras/}}            % pasta de figuras
\usepackage{amssymb}           		 % pacote para fontes e simbolos matemáticos
\usepackage{longtable}          	 % possibilita inserir grandes tabelas
\usepackage{xcolor,colortbl,multirow} % permite textos e tabelas com cores
\usepackage{amsmath}           		 % pacote para equações
\usepackage{epstopdf}
\usepackage{ragged2e}
\usepackage{listings}
%\lstset{numbers=left, numberstyle=\tiny, stepnumber=1, numbersep=5pt, basicstyle=\scriptsize ,frame=tbrl}
\usepackage{etoolbox}
\usepackage{multicol}
%\usepackage{caption}
\usepackage[position=b,singlelinecheck=on]{subfig}

\floatstyle{plaintop}
\newfloat{quadro}{htbp}{loq}
\floatname{quadro}{Quadro}

\makeatletter
\patchcmd{\listof}% <cmd>
{\float@listhead}% <search>
{\@namedef{l@#1}{\l@table}\float@listhead}% <replace>
{}{}% <success><failure>
\makeatother

%\makeatletter
%\renewcommand*{\float@listhead}[1]{%
%	\@ifundefined{chapter}{%
%		\section*{#1}%
%		\addcontentsline{toc}{section}{#1}%
%	}{%
%	\chapter*{#1}% 
%	\addcontentsline{toc}{chapter}{#1}%
%}%
%\@mkboth{\MakeUppercase{#1}}{\MakeUppercase{#1}}%
%}
%\makeatother




%\DeclareFloatingEnvironment{quadro}[Quadro][Lista de quadros]
%\DeclareCaptionType{quadro}[Quadros][Lista de quadros]


% % % % % % % % % % % % % % % % % % % %

\definecolor{codegreen}{rgb}{0,0.6,0}
\definecolor{codegray}{rgb}{0.5,0.5,0.5}
\definecolor{codepurple}{rgb}{0.58,0,0.82}
\definecolor{backcolour}{rgb}{0.95,0.95,0.92}

%\definecolor{verde}{rgb}{0,0,0}

\lstdefinestyle{mystyle}{
	backgroundcolor=\color{backcolour},   
	commentstyle=\color{codegreen},
	keywordstyle=\color{magenta},
	numberstyle=\tiny\color{codegray},
	stringstyle=\color{codepurple},
	basicstyle=\footnotesize,
	breakatwhitespace=false,         
	breaklines=true,                 
	captionpos=b,                    
	keepspaces=true,                 
	numbers=left,                    
	numbersep=5pt,                  
	showspaces=false,                
	showstringspaces=false,
	showtabs=false,                  
	tabsize=2
}

\lstset{style=mystyle}

%	Estrutural
\usepackage{url}
\usepackage{threeparttable}     	 % permite a inserção de notas de rodapé nas tabelas
\usepackage{lscape}             	 % orientação de página LANDSCAPE
\usepackage{pifont}             	 % acrescenta símbolos diferentes.
%\usepackage{cmap}					 % Mapear caracteres especiais no PDF	
\usepackage{lastpage}				 % Usado pela Ficha catalográfica
\usepackage{indentfirst}			 % Indenta o primeiro parágrafo de cada sessão.



% Pacotes de citações
\usepackage[brazilian,hyperpageref]{backref}	 % Paginas com as citações na bibl
\usepackage[num]{abntex2cite}	% Citações padrão ABNT numerica

% remover bordas dos links
\hypersetup{
    colorlinks,
    linkcolor={black},
    citecolor={black},
    urlcolor={black}
}