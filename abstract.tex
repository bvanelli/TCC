\palavraschave{1. SLAM 2. Gmapping 3. Hector 4. Karto 5. Cartographer}  % 5 palavras-chave
\keywords{1. SLAM 2. Gmapping 3. Hector 4. Karto 5. Cartographer}   % 5 palavras-chave para resumo em inglês

\DEELabstract{Robotics has been present in our lives for decades now, largely implemented on industries, but the adoption of robots working closely to humans is still challenging. Although much has been developed in the field of assistive robots, they are still incipient because of all the technology required to interact with users in a meaningful way. This paper aims at discussing a specific task in mobile robots, SLAM, or Simultaneous Localization and Mapping. It comprises the ability of the robot to map unknown environments while having no previous information. The case study will be presented using Care-o-bot, the assistive robot developed at Fraunhofer IPA. Data from laser scanners and odometry is used, and the resulting reconstruction from the most popular algorithms available on ROS (Robot Operating System) will be presented and benchmarked, namely Gmapping, Hector, Karto and Cartographer. Comparisons on mean square error and displacement error will be calculated for each algorithm, as well as proposed calculations for map distortion and CPU and Memory usage.

The results show good stats for Gmapping and Cartographer, some of the most popular choices in the ROS community, Cartographer having the most accurate maps. Hector and Karto seem alternative options for devices with lower computing power, as they can consume far lower CPU on default settings, as well as providing good localization.}

\DEELresumo{A robótica está presente em nossas vidas há décadas, amplamente implementada em indústrias, mas a adoção de robôs trabalhando em colaboração com os seres humanos ainda é um desafio. Embora muito tenha sido desenvolvido no campo dos robôs assistivos, eles ainda são incipientes por causa de toda a tecnologia necessária para interagir com os usuários de maneira significativa. Este artigo tem como objetivo discutir uma tarefa específica em robôs móveis, SLAM, ou Mapeamento e Localização Simultânea. Essa tarefa compreende a capacidade do robô para mapear ambientes desconhecidos sem informações prévias. O estudo de caso será apresentado usando o Care-o-bot, o robô assistivo desenvolvido na Fraunhofer IPA. Dados de scanners a laser e odometria são utilizados, e a reconstrução resultante dos algoritmos mais populares disponíveis no ROS (Robot Operating System) será apresentada e comparada, notavelmente Gmapping, Hector, Karto e Cartographer. Comparações sobre erro quadrático médio e erro de deslocamento serão calculadas para cada algoritmo, bem como os cálculos propostos para distorção do mapa e uso da CPU e da memória.

Os resultados mostram boas estatísticas para Gmapping e Cartographer, escolhas mais populares na comunidade ROS, com o Cartographer tendo os mapas mais precisos. Hector e Karto parecem opções alternativas para dispositivos com menor poder de computação, já que podem consumir muito menos CPU nas configurações padrão, além de fornecer boa localização.}