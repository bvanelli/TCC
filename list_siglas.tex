%----------Lista de Acronimos --------------------------------
\ifx\isenglish\undefined
\pretextualchapter{Lista de Siglas e Abreviaturas}
\else
\pretextualchapter{Acronyms}
\fi


%\renewcommand{\baselinestretch}{2} %espaçamento entre linhas (por definicçào, no espABNT é 2)
\begin{tabbing}
xxxxxxxxxxx \= xxxxxxxxxxxxx \kill
\textsc{ROS}            \> \textit{Robot Operating System}\\
\textsc{UFSC} \> \textit{Universidade Federal de Santa Catarina}\\
\textsc{LIDAR}            \> \textit{Light Detection And Ranging}\\
\textsc{COB}            \> \textit{Care-o-bot}\\
\textsc{SLAM}            \> \textit{Simultaneous localization and mapping}\\
\textsc{XML}            \> \textit{Extensible Markup Language}\\
\textsc{RPC}            \> \textit{Remote Procedure Call}\\
\textsc{URDF}            \> \textit{Unified Robot Description Format}\\
\textsc{SDF}            \> \textit{Simulation Description Format}\\
\textsc{ODE} \> \textit{Open Dynamics Engine} \\
\textsc{DART} \> \textit{Dynamic Animation and Robotics Toolkit} \\
\textsc{IMU} \> \textit{Inertial measurement unit} \\
\textsc{IPA} \> \textit{Institut f{\"u}r Produktionstechnik und Automatisierung} \\
\textsc{DOF} \> \textit{Degree of Freedom} \\
\textsc{API} \> \textit{Application Program Interface} \\
\textsc{ICP} \> \textit{Iterative closest point} \\


\end{tabbing}

%\addcontentsline{toc}{chapter}{Lista de Siglas e Abreviaturas}
